\documentclass{article}
\usepackage[utf8]{inputenc}
\title{Verification: Homework 4}
\author{Marius }
\date{October 2021e}


\usepackage{stmaryrd}	%N
\usepackage{esvect}		%Vector
\usepackage{hyperref}	%hypertext link
\usepackage{graphicx}
\usepackage{amsmath}    %overset
\usepackage{amssymb}    %square
\usepackage{amsthm}		%proof
\usepackage{xcolor}		%color
\usepackage{esvect}		%Vector
\usepackage{tikz}		%draw
\usetikzlibrary{positioning}
\usetikzlibrary{automata}
\usetikzlibrary{arrows}


%%% Operator %%%
\newcommand{\norm}[1]{\left\Vert #1 \right\Vert}
\newcommand{\card}[1]{\ensuremath{\left\|#1 \right\|}}
\newcommand{\interval}[2]{\ensuremath{\llbracket #1, \; #2 \rrbracket}}
\newcommand{\set}[1]{\{ #1 \}}
\newcommand{\br}[1]{\ensuremath{\llbracket #1 \rrbracket}} %Interpretation
\newcommand{\floor}[1]{\lfloor #1 \rfloor}
\newcommand{\ceil}[1]{\lceil #1 \rceil}
\newcommand{\ol}[1]{\overline{#1}}
\newcommand{\ul}[1]{\underline{#1}}

% Shortcuts sets
\newcommand{\bb}[1]{\mathbb{#1}}
\newcommand{\mc}[1]{\mathcal{#1}} %Abrev mathcal
\newcommand{\N}{\ensuremath{\mathbb{N}}}
\newcommand{\Z}{\ensuremath{\mathbb{Z}}}
\newcommand{\Q}{\ensuremath{\mathbb{Q}}}
\newcommand{\C}{\ensuremath{\mathbb{C}}}

%%% shorcuts logic %%%
\newcommand{\G}{\Gamma} % Abreviation pour logique etc..
\newcommand{\D}{\Delta} % Abreviation pour logique
\newcommand{\T}{\mathcal{T}} %Abreviation T pour logique

\newcommand{\V}{\mathcal{V}}
\newcommand{\A}{\mathcal{A}}
\newcommand{\R}{\mathcal{R}}


\begin{document}
\maketitle
\noindent
$\bullet$ $\br{p}= \{1,2,4,5\}$\newline
$\bullet$ $\br{EGp}= \{1,2,4,5\}$ respectively with paths $(12)^\omega,(21)^\omega,45^\omega,5^\omega$\newline
$\bullet$ $\br{q}= \{2,3,4\}$\newline
$\bullet$ $\br{EXq}= \{0,1,2,3,4\}$ respectively because $0 \rightarrow 3, 1 \rightarrow 2, 2 \rightarrow 3$ and $3 \leftrightarrow 4$\newline
$\bullet$ $\br{AGEXq}= \emptyset$ because 5 is reachable from every other states\newline
$\bullet$ $\br{E((EGp)U(AGEXq)}= \emptyset$ as a direct consequence.\newline
$\bullet$ $\br{EXEGp}= \{0,1,2,3,4,5\}$ because $0 \rightarrow 3, 1 \leftrightarrow 2, 2 \rightarrow 3, 3 \leftrightarrow 4$ and $5 \rightarrow 5$\newline
$\bullet$ $\br{A((EXEGp)Uq)}= \{0,1,2,3,4\}$ because $5^\omega$ never satisfies $q$. \newline	



\newpage


\end{document}
