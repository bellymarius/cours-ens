\documentclass[pdf]{beamer}
\usepackage[utf8]{inputenc}
\usefonttheme{professionalfonts}
\usetheme{Madrid}
\useoutertheme{shadow}
\useinnertheme{circles} %Less sharpened edges
\usecolortheme{}
\title{Verification : Summary}
\author{Marius}
\date{07/01}

\usepackage{stmaryrd}   %N
\usepackage{esvect}     %Vector
\usepackage{hyperref}   %hypertext link
\usepackage{graphicx}
\usepackage{amsmath}    %overset
\usepackage{amssymb}    %square
\usepackage{amsthm}     %proof
\usepackage{xcolor}     %color
\usepackage{esvect}     %Vector
\usepackage{tikz}       %draw
\usetikzlibrary{positioning}
\usetikzlibrary{automata}
\usetikzlibrary{arrows}


%%% Operator %%%
\newcommand{\norm}[1]{\left\Vert #1 \right\Vert}
\newcommand{\card}[1]{\ensuremath{\left\|#1 \right\|}}
\newcommand{\interval}[2]{\ensuremath{\llbracket #1, \; #2 \rrbracket}}
\newcommand{\set}[1]{\{ #1 \}}
\newcommand{\br}[1]{\ensuremath{\llbracket #1 \rrbracket}} %Interpretation
\newcommand{\floor}[1]{\lfloor #1 \rfloor}
\newcommand{\ceil}[1]{\lceil #1 \rceil}
\newcommand{\ol}[1]{\overline{#1}}
\newcommand{\ul}[1]{\underline{#1}}

% Shortcuts sets
\newcommand{\bb}[1]{\mathbb{#1}}
\newcommand{\mc}[1]{\mathcal{#1}} %Abrev mathcal
\newcommand{\N}{\ensuremath{\mathbb{N}}}
\newcommand{\Z}{\ensuremath{\mathbb{Z}}}
\newcommand{\Q}{\ensuremath{\mathbb{Q}}}
\newcommand{\C}{\ensuremath{\mathbb{C}}}

%%% shorcuts logic %%%
\newcommand{\G}{\Gamma} % Abreviation pour logique etc..
\newcommand{\D}{\Delta} % Abreviation pour logique
\newcommand{\T}{\mathcal{T}} %Abreviation T pour logique

\newcommand{\M}{\mathcal{M}}
\newcommand{\V}{\mathcal{V}}
\newcommand{\A}{\mathcal{A}}
\newcommand{\R}{\mathcal{R}}

\begin{document}
\maketitle
\begin{frame}
\frametitle{LTL and FOL}

\begin{block}{Definition}
\textbf{TL(AP,SU,SS)} is a temporal logic with all the past and future quantifiers.\newline
\textbf{FO(AP,<)} is the first-order logic.
\end{block}


\begin{exampleblock}{Intuition}
$w,i \models \phi \mathbf{SS}\psi \Leftrightarrow $ Il y a eu un instant passé auquel $\psi$ était vrai et après lequel $\phi$ était vrai jusqu'à aujourd'hui.\newline
$w,i \models \phi \mathbf{SU} \psi \Leftrightarrow $ Il y aura un instant futur auquel $\psi$ sera vrai et jusqu'auquel $\phi$ sera vrai.\newline
Les operateurs \textbf{SS} and \textbf{SU} permettent de recréer tout les quantificateurs usuels.
\end{exampleblock}


\begin{alertblock}{Theorems}
TL $\subseteq$ FO [Using the model $\N$]\newline
\end{alertblock}

\end{frame}

\begin{frame}
\frametitle{CTL and MSO}

\begin{block}{Definition}
\textbf{CTL$^*$(AP,SU)} is a temporal logic enabling path quantifiers \textbf{E} and \textbf{A} .\newline
\textbf{CTL(AP,SU)} is a temporal logic resticting $CTL^*$ where all path quantifiers are followed by exactly one temporal operator.\newline
\textbf{MSO(AP,<)} is the monadic second-order logic [enabling quantifying over sets]
\end{block}


\begin{exampleblock}{Intuition}
$\M,\lambda \models \phi  \Leftrightarrow \phi$ est vrai dans le chemin $\lambda$.
\end{exampleblock}


\begin{alertblock}{Theorems}
CTL $\subseteq$ MSO [Using the model $\N$]\newline
CTL$^*$ $\subseteq$ MSO [Using the model $\N$]\newline
\end{alertblock}

\end{frame}


\begin{frame}
\frametitle{Buchi Automata}

\begin{block}{Definition}
\textbf{Büchi-automaton} is NFA where a $\omega$-word is recognized if it goes infinitely many times through $F$.\newline
\textbf{General Büchi-automaton} is a BA where a $\omega$-word is recognized if it goes infinitely many times through all the $F_i$.\newline
\end{block}




\begin{alertblock}{Some properties}
Büchi-automata are closed under boolean operations.\newline
[intersection : go from a copy of the cartesian product to another when meeting a target state] \newline
LTL \subseteq L(BA) \newline
Büchi-automata and General Büchi-automata where are the same \newline	
Büchi-automata and Deterministic Büchi-automata are NOT the same \newline	

CTL$^*$ $\subseteq$ MSO [Using the model $\N$]\newline
\end{alertblock}

\end{frame}

\begin{frame}
\frametitle{Partial Order Reduction : Definition}

\begin{block}{Definition}
A \textbf{labelled Kripke structure} is extended with actions.\newline
An \textbf{independence relation} over actions is irreflexive, symmetric and confluent [if both actions can be chosen, they can be done in both orders]\newline
A set of actions is\textbf{invisible} if it never change the truth of a AP.\newline
The \textbf{stuterring-equivalence} of sequences is "equality modulo multiplicity"
\end{block}


\begin{alertblock}{Some properties}
Any LTL without \textbf{X} is invariant under stuterring.
\end{alertblock}
\end{frame}

\begin{frame}
\frametitle{Partial Order Reduction : General Idea}
\begin{exampleblock}{Ample set method}
Idea : build a set of explored actions $red(s)$ following conditions :\newline
1. Keep at least one action by state when possible [Avoid adding deadlocks] \newline
2. Actions depending on $red(s)$ occurs after $red(s)$ in a concrete path starting with $s$. \newline
2bis. In particular, all actions not from $red(s)$ are independent with all actions from $red(s)$ \newline
3. Keep all actions or only an invisible subset [Preserve stuttering equivalence]. \newline
4. An abstract action in a state $s$ in a cycle is in $red(cycle)$ [No starving]. \newline	
\end{exampleblock}
\end{frame}


\begin{frame}
\frametitle{Fair Kripke structure and complexity}
\begin{block}{Definition}
A  \textbf{fair kripke structure} is extended states set visited infinitely often. \newline	
Can be simulated in CTL$^*$ with $\bigwedge GF F_i$ \newline
Cannot be simulated in CTL. \newline

\end{block}

\begin{alertblock}{Complexity}
LTL satisfiability is \textbf{PSPACE}-complete. \newline
CTL$^*$ model checking is \textbf{PSPACE}-complete. \newline
CTL model checking is decidable in time $O(\vert M \vert \cdot \vert \phi \vert)$
CTL model checking is decidable in time $O(\vert M \vert \cdot \vert \phi \vert)$
\end{alertblock}

\end{frame}

\begin{frame}
\frametitle{Binary Decision Diagrams}
\begin{block}{Definition}
A \textbf{Binary Decision Graph} is a DAG labelling by variables and stricty ordering them. Leaves are labelled 0 and 1.\newline
A \textbf{Binary Decision Diagram} is a BDG with no subgraphs isomorphic and no redundant nodes.\newline
A \textbf{Binary Decision Diagram with complement arcs} is a BDD with potential filled circle on edges inverting the meaning\newline
\end{block}

\begin{exampleblock}{Simplification}
One can merge nodes with the same successors and neglect the leaf labelled 0.
\end{exampleblock}

\begin{alertblock}{Theorem}
Given a function $\V \rightarrow \bb{B}^n$, there is an unique BDD up to isomorphism. \newline
\end{alertblock}

\end{frame}

\begin{frame}
\frametitle{Binary Decision Diagrams : Operations}

\begin{alertblock}
Equivalence problem can be solved through isomorphic test \newline
Negation can be solved through exchange of leaves \newline
Conjonction can be solved through the following formula : \newline
$ite(x,F[x:=1]\wedge G[x:=1],F[x:=0]\wedge G[x:=0])$ \newline
Disjonction can be solved in the same way \newline
CBDD are unique only if we prohibit negated 0-labelled edges\newline
\end{alertblock}

\end{frame}

\begin{frame}
\frametitle{Pushdown Systems}

\begin{block}{Definition}
A \textbf{Pushdown system} is basically the transition system of a nested stack automaton without any input.\newline
The \textbf{P-Automaton} is the true notion of automaton with $qw$ as inputs.\newline
\end{block}

\begin{exampleblock}{Intuition}
Replacing the last item pushed $\sim$ classical instruction \newline
Pushing a new item $\sim$ procedure call \newline
Poping the last item $\sim$ return
\end{exampleblock}

\begin{alertblock}{Property}
Reachability from a configuration to another is decidable
\end{alertblock}

\end{frame}

\begin{frame}
\frametitle{Petri Nets}

\begin{block}{Definition}
A \textbf{Petri Net} is a tuple $\lbracket P,T,F,W,wm_0 \rbracket$, where : \newline
$\bullet$ $P$ is the set of places \newline
$\bullet$ $T$ is the set of transitions \newline
$\bullet$ $F \subseteq P \times T \cup T \times P$ is the flow relations \newline
$\bullet$ $T$ is the set of transitions \newline

The \textbf{P-Automaton} is the true notion of automaton with $qw$ as inputs.\newline
\end{block}

\begin{exampleblock}{Intuition}
Replacing the last item pushed $\sim$ classical instruction \newline
Pushing a new item $\sim$ procedure call \newline
Poping the last item $\sim$ return
\end{exampleblock}

\begin{alertblock}{Property}
Reachability from a configuration to another is decidable
\end{alertblock}

\end{frame}

\end{document}
