\documentclass[pdf]{beamer}
\usepackage[utf8]{inputenc}
\usefonttheme{professionalfonts}
\usetheme{Madrid}
\useoutertheme{shadow}
\useinnertheme{circles} %Less sharpened edges
\usecolortheme{}
\title{Computer Vision : Summary}
\author{Marius }
\date{October 2021}

\usepackage{stmaryrd}	%N
\usepackage{esvect}		%Vector
\usepackage{hyperref}	%hypertext link
\usepackage{graphicx}
\usepackage{amsmath}    %overset
\usepackage{amssymb}    %square
\usepackage{amsthm}		%proof

\newcommand{\V}{\mathcal{V}}
\newcommand{\N}{\mathbb{N}}
\newcommand{\R}{\mathcal{R}}
\renewcommand{\O}{\mathcal{O}}

\begin{document}
\maketitle


\begin{frame}
\frametitle{Edge shape detection}

\begin{block}{RANSAC}
$\bullet$ Select a few points \newline
$\bullet$ Least square regression \newline
$\bullet$ Check wether the other points fit the regression \newline
\end{block}


\begin{block}{Hough Transform}
Parameter space : (ordinate at origin, angle)
Intensity of a point proportional to the nummber of points it explains.
\end{block}

\end{frame}
\begin{frame}
\frametitle{Fourier Transform}


\begin{block}{Fourier Transform}
The image is a signal, thus a sum of sine waves. \newline
Technically two images (frequence and phase) but we often omit the latter.\newline
The pixel $(f,\theta)$ has an intensity proportionnal to the coefficient of the sine wave of frequency $f$ in direction $\theta$.\newline
The Fourier transform is always symmetrical as the left part is a reflect of the right one. (origin is the center of the image)
\end{block}

\end{frame}
\begin{frame}
\frametitle{Fourier Transform (II)}

\begin{alertblock}{Invertibility and Compatibility}
The Fourier transform is invertible (no loss of information) \newline
Fourier transform (and its invert) is commutative with convolutions. \newline
Convolutions of two images matches element-wise product in the Fourier space\newline
\end{alertblock}

\begin{exampleblock}{Fast Fourier Transform}
This enables an efficient algorithm ($\O(n \log n)$) to apply convolutions using a "divide to conquer" paradigm.
\end{exampleblock}


\end{frame}

\begin{frame}
\frametitle{Fourier Transform : Applications}

\begin{exampleblock}{Low/High-Filter}
Apply a filter to keep only low/high frequencies in the fourier transform.
\end{exampleblock}


\begin{exampleblock}{Debluring}
$\bullet$ No noise : Division in Fourier Space, only if the kernel is never null.\newline
$\bullet$ Noise : Wiener Filter
\end{exampleblock}

\end{frame}

\begin{frame}
\frametitle{Bilateral filters}


\begin{exampleblock}{Gaussian smoothing : backdraws}
Gaussian smoothing are blur filters where dependencies between pixels intensities increase with their physical proximity. \newline
This may blur edges because two neighbours separated by an edge will tend to converge one to another.
\end{exampleblock}

\begin{block}{Bilateral filter}
Bilateral filter uses a gaussian smoothing $G_s$ along with another gaussian smoothing $G_r$ using the difference in intensity as the distance.\newline
As a result, dependencies between pixels intensities increase both with their physical proximity and their intensity proximity. \newline
The coefficient are adjusted so that a pixel is always a barycenter of the pixels in the initial image.\newline
The resulting image is smooth (details were removed) while preserving edges.
\end{block}

\end{frame}

\begin{frame}
\frametitle{Bilateral filters : Remarks}


\begin{exampleblock}{Implementation tricks}
Bilateral filter are not easily computable (cannot use FFT).\newline
However, as Gaussian filters are low-pass, we may only compute the $G_r$ on close neighbours. 
\end{exampleblock}

\begin{exampleblock}{Details enhancement}
On the contrary, if we want to enhace details : \newline
$\bullet$ Compute $I = BF(I)+D(I)$ \newline
$\bullet$ Compute $I' = BF(I)+k \times D(I)$ \newline
\end{exampleblock}

\begin{exampleblock}{Cross-Bilateral Filter}
This technique may be used on a pair $(I,E)$, where $E$ has been taken with flash.\newline
We use $E$ to determine intensities proximity and apply bilateral filter on $I$.\newline
The resul is a denoised image without the flash.
\end{exampleblock}

\end{frame}

\begin{frame}
\frametitle{HDH Images}

\begin{block}{High Dynamic Range Images}
Same scene captured with different exposure shots\newline
Each shot typically captures well only a part of the scene. \newline
Empirically, linear mapping does not work well to combine them (low-intensities pixels prevail). \newline
\end{block}

\begin{exampleblock}{Contrast and model}
A contrast is given by an increasing function $f I \rightarrow  f \circ I$ \newline
Linear contrast are exposure times changes ! \newline
\end{exampleblock}

\end{frame}

\begin{frame}
\frametitle{HDH Images (II)}

\begin{block}{Intensity histogram and cumulative histogram}
$H_I(i) = \# \{x \vert I(x) = i \}$ \newline
$C_I(i) = \# \{x \vert I(x) < i \}$ \newline
\end{block}

\begin{exampleblock}{Interest of $C$}
As $C$ is increasing, $C_{f \circ I} = C_I \circ f$ \newline
In practice : $f = C^{-1}_{target} \circ C_I$ \newline
First, we separate large scale from details using BF.\newline
We may then for instance homogenise the picture separately \newline
 $\rightarrow$ constant $H_{target}$ \newline
\end{exampleblock}

\end{frame}

\begin{frame}
\frametitle{Colors}

\begin{block}{Human perception}
A \emph{cone} is basically a filter on the spectrum.\newline
A \emph{metamere} is a collision in our perception.\newline
RGB seems sufficient to match any perceptible color if we can add them to both side [encoded as subtractive matching]
\end{block}

\begin{alertblock}{Empirical Grassman's law}
Color matching appears to be linear
\end{alertblock}

\begin{alertblock}{A fundamental assertion}
I am really not interested in white balance.
\end{alertblock}

\end{frame}

\begin{frame}
\frametitle{Projective Geometry}

\begin{block}{Projective space}
A \emph{projective space} is the set of equivalence classes for the relation "being colinear" in a vector space.\newline
Basically, those are the lines passing by 0. \newline
The projective 3D-space can be seen as the plane $z=1$ (each point is the projection of a line)...\newline
... to which we add the line $\{y=1,~z=0\}$ (projection of lines in the plane $z=0$)... \newline
...to which we add a $\infty$-point (projection of the line $\{y=0,z=0\}$) \newline
\end{block}

\end{frame}
\end{document}
