\documentclass[pdf]{beamer}
\usepackage[utf8]{inputenc}
\usefonttheme{professionalfonts}
\usetheme{Madrid}
\useoutertheme{shadow}
\useinnertheme{circles} %Less sharpened edges
\usecolortheme{}
\title{Logic for AI : Part II}
\author{Marius}
\date{December 2021}

\usepackage{stmaryrd}   %N
\usepackage{esvect}     %Vector
\usepackage{hyperref}   %hypertext link
\usepackage{graphicx}
\usepackage{amsmath}    %overset
\usepackage{amssymb}    %square
\usepackage{amsthm}     %proof
\usepackage{xcolor}     %color
\usepackage{esvect}     %Vector
\usepackage{tikz}       %draw
\usetikzlibrary{positioning}
\usetikzlibrary{automata}
\usetikzlibrary{arrows}


%%% Operator %%%
\newcommand{\norm}[1]{\left\Vert #1 \right\Vert}
\newcommand{\card}[1]{\ensuremath{\left\|#1 \right\|}}
\newcommand{\interval}[2]{\ensuremath{\llbracket #1, \; #2 \rrbracket}}
\newcommand{\set}[1]{\{ #1 \}}
\newcommand{\br}[1]{\ensuremath{\llbracket #1 \rrbracket}} %Interpretation
\newcommand{\floor}[1]{\lfloor #1 \rfloor}
\newcommand{\ceil}[1]{\lceil #1 \rceil}
\newcommand{\ol}[1]{\overline{#1}}
\newcommand{\ul}[1]{\underline{#1}}

% Shortcuts sets
\newcommand{\bb}[1]{\mathbb{#1}}
\newcommand{\mc}[1]{\mathcal{#1}} %Abrev mathcal
\newcommand{\N}{\ensuremath{\mathbb{N}}}
\newcommand{\Z}{\ensuremath{\mathbb{Z}}}
\newcommand{\Q}{\ensuremath{\mathbb{Q}}}
\newcommand{\R}{\ensuremath{\mathbb{R}}}

%%% shorcuts logic %%%
\newcommand{\G}{\Gamma} % Abreviation pour logique etc..
\newcommand{\D}{\Delta} % Abreviation pour logique
\newcommand{\T}{\mathcal{T}} %Abreviation T pour logique

\newcommand{\V}{\mathcal{V}}
\newcommand{\A}{\mathcal{A}}
\newcommand{\F}{\mathcal{F}}

\begin{document}
\maketitle

\begin{frame}
\frametitle{Davis-Putman-Loveland-Logmann}

\begin{block}{DPLL Algorithm}
$\bullet$ Guess the assignment of a variable \newline
$\bullet$ Propagate the decision through unit propagation \newline
$\bullet$ In case of conflict, switch the last decision\newline
\end{block}

\end{frame}

\begin{frame}
\frametitle{Conflit Driven Clause Learning}

\begin{block}{CDCL Algorithm}
$\bullet$ Guess the assignment of a variable \newline
$\bullet$ Propagate the decision through unit propagation \newline
$\bullet$ In case of conflict, switch the last decision and add a cause preventing the partial assignment that led to the conflict. \newline
\end{block}

\end{frame}


\begin{frame}
\frametitle{Conflit Driven Clause Learning II}

\begin{block}{Definitions}
A \emph{cut clause} is a cut that separates the decisions from the conflict in an oriented graph where edges are (partial) implication of unit propagation. \newline
A \emph{Unique Implication Point (UIP)} is a literal $l$ propagated at decision level $d$ occuring in each path from decision $d$ to the conflict  \newline
\end{block}

\begin{exampleblock}{Optimisations}
Any cut clause may replace the clause learnt.\newline
A way to do it is to iteratively replace a decision with an UIP, preferably the closest one to the conflict \newline
Then, We may add \emph{back clauses} to keep informations about the relations between UIPs and decisions.\newline
\end{exampleblock}
\end{frame}

\begin{frame}
\frametitle{Conflit Driven Clause Learning in theory $\T$}

\begin{block}{Definitions}
A \emph{boolean encoder} $e$ replaces atoms of the theories with new boolean \newline 
(e.g $x<2 \rightarrow x_0$)\newline
A \emph{unsat core} is a minimal subset of the atoms of the formula that is unsat.\newline
\end{block}

\begin{block}{CDCL($\T$) Algorithm}
$\bullet$ Apply CDCL Algorithm \newline
$\bullet$ At each decision, apply Theory Propagation Algorithm.\newline
\end{block}


\begin{block}{Theory Propagation Algorithm}
$\bullet$ Check Satisfiability of $e^{-1}(m)$ \newline
$\bullet$ If UNSAT, return an unsat core and a decision level.\newline
$\bullet$ If SAT returns atoms of the input formula implied by $e^{-1}(m)$.\newline
\end{block}


\end{frame}

\begin{frame}
\frametitle{Theory 1 : Equality and Function symbols (basics)}

\begin{block}{EUF}
Signature : $\{\F,=\}$ \newline
Deduction : Symmetry, Transitivity, Congruence
\end{block}

\begin{exampleblock}{Remark}
Constant may be removed (There is an equisatisfiable formula).
\end{exampleblock}

\end{frame}

\begin{frame}
\frametitle{Theory 1 : Equality and Function symbols (TheoryPropagation)}

\begin{block}{Ackermann's reduction}
$\bullet$ Encode terms with new constants\newline
$\bullet$ Encode atoms with variables (boolean encoder) \newline
$\bullet$ Solve the satisfiability of $F \wedge Rules$ \newline
This is possible because applicable rules are finite and determined by the subterms of the initial formula.
\end{block}

\begin{exampleblock}{A quick example}
$x=y \Rightarrow f(x)=f(y)\newline
t_1=t_2 \Rightarrow t_3=t_4, t_1=t_2 \Rightarrow t_2=t_1$... \newline 
$p_1 \Rightarrow p_2, p_1 \Rightarrow p_3$...
\end{exampleblock}

\end{frame}

\begin{frame}
\frametitle{Theory 1 : Equality and Function symbols (	TheoryPropagation II)}

\begin{block}{Lazy-QF\_EUF}
$\bullet$ Process each subformula $t_1 \otimes t_2$.\newline
$\bullet$ Keep tracks of equivalence class among terms and subterms.\newline
\end{block}

\begin{exampleblock}{A quick example}
$f(x)=x \wedge f(f(x)) \neq x \newline
f(x)=x \rightarrow C=[\{x,f(x)\}]\newline
f(f(x)) \neq x \rightarrow C=[\{x,f(x)\},\{f(f(x))\}],D=[(C_0,C_1)]\newline
f(x),f(f(x)) \in T \wedge x \sim f(x) \rightarrow \text{ merge } C_0 \text{ and } C_1$. \newline
As $(C_0,C_1) \in D$, return UNSAT.
\end{exampleblock}

\end{frame}


\begin{frame}
\frametitle{Theory 2 : Linear Rational Arithmetic}

\begin{block}{LRA}
Signature : $\{\N,<,+\}$ \newline
Deduction : Non-negative linear combination of inequalities
\end{block}

\begin{alertblock}{Fundamental Theorem of Linear Inequality}
Let $(a_i)_{0\leqi<m}$ and $b \in \R^n$. Then : \newline
$\bullet$ Either $ \exists \lambda_0...\lambda_{m-1} ~b=\lambda_0 a_0...\lambda_{m-1} a_{m-1}$ and $a_1...a_m$ linearly independent. \newline
$\bullet$ Or there exists a hyperplane $\{c \vert cx=0\}$ such that $ca_i \geq0$ and $cb<0$
\end{alertblock}

\begin{exampleblock}{Remarks}
This is a fancy way to say that either $b$ is in the cone generated by $(a_i)_{0\leq i<m}$ or it is not.\newline
A \emph{cone} generated by $(a_i)_{0\leq i<m}$ is $\{\lambda_0 a_0...\lambda_{m-1} a_{m-1}\}$
\end{exampleblock}

\end{frame}

\end{document}